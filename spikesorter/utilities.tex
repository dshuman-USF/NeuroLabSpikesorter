\documentclass{article}
\usepackage{changepage}
\usepackage{graphicx}
\usepackage{natbib}
\usepackage[bookmarks=true,colorlinks=true,linktoc=page,linkcolor=blue,citecolor=blue]{hyperref}
\usepackage{amsmath}
\usepackage{subfigure}
\usepackage{comment}
\usepackage{upquote}
\parindent0pt \parskip8pt
\begin{document}

\title{utilities}
\author{Russell O'Connor}
\maketitle
\clearpage

\section{\tt crossings2edt}
The {\tt crossings2edt} program generates an {\tt .edt} file from a
{\tt .chan} file, based on positive-going crossings of a specified
threshold.  This can be useful if there is one large, clean unit on
the channel.  One spike is placed in the {\tt .edt} file for each
place in the {\tt .chan} file where there is a sample below threshold
followed by zero or more samples at threshold, followed by a sample
above threshold.  The spike time is the time of the above-threshold
sample rounded to the nearest .1 ms.  Running the program with no
arguments gives this usage message:
\begin{verbatim}
usage: crossings2edt whatever.chan threshold unit_code
\end{verbatim}
{\tt whatever.chan} is the input {\tt .chan} file.  The output {\tt
  .edt} file will have the same base name, but with an {\tt .edt}
extension.  Any existing file by that name will be overwritten without
warning.  {\tt threshold} is a number between -32768 and 32767
chosen by the user, typically by reading it off the waveform display
(using {\tt waveform.tcl}) as the fifth number displayed at the top,
when the cursor is positioned at what looks like an appropriate
threshold level.  {\tt unit\_code} is the cell ID that will be used for
the spike train in the {\tt .edt} file.

\section{\tt crossings2pos}
The {\tt crossings2pos} program generates a {\tt .pos} file with spike
times for use with the spikesorter, from a {\tt .chan} file, based on
positive-going crossings of a specified threshold.  This can be useful
if there is one large, clean unit on the channel.  Running it with no
arguments gives this usage message:
\begin{verbatim}
usage: crossings2pos [-m THR2] POS CHAN THR CLUSTER... > OUTPOS
\end{verbatim}
The program is intended to be used with a {\tt POS} file that has
been generated by the spikesorter from {\tt CHAN}, but for which
simple threshold crossings would have done a better job. {\tt
  THR} is a number between -32768 and 32767 chosen by the user,
typically by reading it off the waveform display (using {\tt
  waveform.tcl}) as the fifth number displayed at the top, when the
cursor is positioned at what looks like an appropriate threshold
level.  {\tt CLUSTER...} is a space-separated list of cluster numbers
in {\tt POS} that will be left out of {\tt OUTPOS}.  The first
number in the list will be re-used in {\tt OUTPOS} as the cluster
number for the threshold-crossing spikes.  If the old {\tt .pos} file
is deleted or renamed or moved, the new {\tt .pos} file can be given
the name of the old one, and then it can be viewed in the spikesorter.
The waveform overlay for the new cluster will be wrong, as will
scatterplots that include it, but everything else should work.

If a second threshold is specified with the {\tt -t} option, a spike
will not be placed at a positive-going crossing of the first threshold
unless the signal in CHAN goes back below the first threshold without
going above the second threshold.

An example session might look like this:
\begin{verbatim}
cd 2004-01-25_001
crossings2pos 2004-01-08_001_11.pos 2004-01-08_001_11.chan 28000 1 2 3 4 5 > new.pos
mkdir -p save
mv -i 2004-01-08_001_11.pos save
mv new.pos 2004-01-08_001_11.pos
\end{verbatim}

\section{\tt dmx}
The {\tt dmx} program is used by {\tt spikesort\_control\_panel} to
determine the channel labels and which ones are enabled, but it can
also be used by itself on the command line.

Invoked as
\begin{verbatim}
dmx DMXFILE > OUTFILE
\end{verbatim}
where DMXFILE is a DataMAX header or recording file, it will translate
the header to text and write it to OUTFILE.

Invoked as
\begin{verbatim}
dmx DMXFILE labels
\end{verbatim}
it will print a list of the channel labels to the screen.

Invoked as
\begin{verbatim}
dmx DMXFILE enabled
\end{verbatim}
it will print a list of zeroes and ones to the screen, indicating
which channels are enabled (1 = enabled).

If DMXFILE is a recordong file, invoking it as
\begin{verbatim}
dmx DMXFILE N
\end{verbatim}
will extract channel N (the first channel is channel 1) from DMXFILE
and write it to a {\tt .chan} file in the current directory.  The name
of the {\tt .chan} file will be the same as DMXFILE with any extension
deleted, and with a two-digit channel number and the {\tt .chan}
extension appended.  Andy existing file by that name will be
overwritten without warning.

\section{\tt dmx\_split}
The {\tt dmx\_split} programs splits a DataMAX recording into {\tt
  .chan} files.  When invoked without arguments, it gives this usage
message:
\begin{verbatim}
usage: dmx_split dmx_file_name
\end{verbatim}
{\tt dmx\_split} creates two directories named {\tt clean} and {\tt split} in
the current directory if they don't already exist.  For each enabled
channel, it will create a {\tt .chan} file in the {\tt split}
directory.  The name of the {\tt .chan} file will be the same as
{\tt dmx\_file\_name} with periods changed to underscores, and with an underscore, a
two-digit channel number, and the {\tt .chan} extension appended.  The
first channel is channel 1 (not 0).  If any of the {\tt .chan} files
already exist, {\tt dmx\_split} will ask once whether to overwrite
them.  If the answer is no, it will exit immediately and no {\tt
  .chan} files will have been written.

It will also create a header file with the same name as {\tt
  dmx\_file\_name}, but in the {\tt clean} directory, containing just
the header from {\tt dmx\_file\_name}.  If the file already exists, it
will be overwritten without warning.  If you answered no to
overwriting {\tt .chan} files as described above, the header file will
still be overwritten if it exists, but the new file will be empty.

\section{\tt info\_file.pl}
The {\tt info\_file.pl} script reads a coordinate spreadsheet file and
a {\tt .nam} file and writes an ``info'' spreadsheet file.  It is
normally invoked automatically by \verb#spikesort_control_panel# after
the user clicks on the MERGE button, so the user doesn't need to know
about it, but it can be invoked by itself from the command line if the
user wants to recreate the info file without rerunning the merge.
When invoked without arguments, it gives this usage message:
\begin{adjustwidth}{-8em}{}
\begin{verbatim}

usage: info_file.pl datadir prefix

info_file.pl will look for a coordinate spreadsheet file at /oberon/experiments/prefix/prefix.xls
info_file.pl will look for a .nam file at datadir/prefix.nam
info_file.pl will  write  an info file at /oberon/experiments/prefix/prefix_info_orig.xls

\end{verbatim}
\end{adjustwidth}
The coordinate spreadsheet file must be in Excel 95-2003 format, and
the output info file will be as well.  The coordinate spreadsheet is
created by hand.  The {\tt .nam} file is generated by
\verb#spikesort_control_panel# after the user clicks the MERGE
button. (\verb#spikesort_control_panel# calls \verb#merge_edt# which
calls  {\tt info\_file.pl}.)

The {\tt .nam} file is an ASCII text file with Linux-format line
endings (line feed).  It has one line per cell, with three items per
line:
\begin{enumerate}
\item the digital channel number
\item the cell name assigned by \verb#edt_merge#
\item the merged channel number assigned by \verb#edt_merge#
\end{enumerate}

{\tt info\_file.pl} requires the coordinate spreadsheet to have the
following exact text in the indicated cells:
\begin{quote}
\begin{description}
\item[Cell C11:] DIGITAL CHANNEL
\item[Cell I11:] A/P
\item[Cell J11:] R/L
\item[Cell K11:] DEPTH
\item[Cell A3:] EXPERIMENT:
\item[Cell A5:] RECORDING:
\item[Cell A7:] DATE:
\end{description}
\end{quote}
There can be a second column of data with the same headers in the same
rows, but starting in column P.  If there is no second column, the
header cells must be empty.

The data must start in row 12, and the background of the data cells
must be colored.  The row following the data must not be colored.

{\tt info\_file.pl} also requires the following information in the
indicated cells:
\begin{quote}
\begin{description}
\item[Cell D48:] the digital channel number of the reference electrode
\item[Cell D50:] the A/P coordinate of the reference electrode
\item[Cell D51:] the R/L coordinate of the reference electrode
\item[Cell D52:] the depth coordinate of the reference electrode
\end{description}
\end{quote}

At USF, the coordinate spreadsheet is generated from a template in the
Excel workbook found from Windows at\\*
\verb#\\cisc3\experiments\templates\coordinate_template_workbook.xls#
(or at
\verb#/oberon/experiments/templates/coordinate_template_workbook.xls#
from Linux).  Instructions can be found at\\*
\verb#\\cisc3\experiments\templates\coordinate_template_instructions.doc#.

The output info file for each experiment contains the name and date of
the experiment; the recording number; the names, channel numbers, and
coordinates of each cell; and places for the results of AA and STA
testing.  The data in the info file is read into {\tt xanalysis} for
display and, from there, is written out for import into a database (at
USF, that's the GAIA database).

{\tt info\_file.pl} makes two copies of the output file, one with a
name of the form \emph{prefix}\verb#_info_orig.xls# as in the usage
message, and the other with a name of the form
\emph{prefix}\verb#_info_curr.xls#.  If
\emph{prefix}\verb#_info_orig.xls# already exists, it is overwritten
without warning.  If \emph{prefix}\verb#_info_curr.xls# already
exists, a warning is written to the command line and the old version
is backed up to a file with a numbered extension, starting with
\emph{prefix}\verb#_info_curr.xls.~1~#.  No existing backups are
overwritten or deleted.  If AA or STA information was written to the
old \emph{prefix}\verb#_info_curr.xls#, it is copied to both new
files.

The two arguments to {\tt info\_file.pl} tell it where to find the
input files and where to put the output file as shown in the usage
message above.  (The \\*{\tt /oberon/experiments} directory shown in
the usage message is the path as seen from Linux.  As seen
from the Windows systems at USF, the same directory is 
\verb# \\cisc3\experiments#.)  It does not matter what the current
working directory is when {\tt info\_file.pl} is invoked.

If the user wants to run {\tt info\_file.pl} without
writing the output to the official directory, the user can specify a
different directory as a third argument, and the output file will be
written there instead.

\section{\tt integrate}
The {\tt integrate} program is used by {\tt spikesort\_control\_panel}
to process channels for which ``I'' is specified, but it can also be
used by itself on the command line.  Invoke it as
\begin{verbatim}
usage: integrate FILE.chan ANALOG_ID [CUTCODE CUT_EDT [flip]]
\end{verbatim}
to rectify and integrate the signal in {\tt FILE.chan} and write the
result to {\tt FILE.edt} with the specified {\tt ANALOG\_ID}.  Also
writes {\tt FILE.bin}, which is a big-endian copy of {\tt FILE.chan}.
If {\tt FILE.edt} or {\tt FILE.bin} exists, it will be overwritten.
Also appends an {\tt I} to {\tt FILE.status}.

If {\tt CUTCODE} and {\tt CUT\_EDT} are specified, regions will be
left out of the integrated signal in {\tt FILE.edt}.  The regions to
be left out are specified by an event code in an {\tt .edt} file.
{\tt CUTCODE} is the id of the event code, and {\tt CUT\_EDT} is the
name of the {\tt .edt} file.  The region between the first and second
event of the given id is left out, and between the third and fourth,
etc.  In general, the region between each odd numbered event (start
marker) and the following even numbered event (stop marker) is left
out.  If ``flip'' is specifed, the regions that would have been left
out are kept, and vice-versa. The content of the left-out regions has
no effect on the integrated signal.

Before each kept region, the integrator is initialized with the mean
of the integrated signal, so there is little or no start-up ramp.

The samples of the integrated signal are 5 milliseconds apart, and the
sampling clock continues to run during the left-out regions, so all
intervals between samples are a multiple of 5 ms.

\section{\tt split\_merge\_abf}
The {\tt split\_merge\_abf} programs merges {\tt .abf} files and then
splits them into {\tt .chan} files.  When invoked without arguments,
it gives this usage message:
\begin{verbatim}
usage: split_merge_abf whatever.abf [channel]...
\end{verbatim}
The {\tt whatever.abf} file must end in a {\tt .abf} extension, but
the {\tt .abf} can be upper or lower case (or even mixed).  The file
is assumed to have a 2048 byte header, which is ignored, followed by
consecutive interleaved 16-bit samples from 16 channels in channel
order 1,9,5,13,2,10,6,14,3,11,7,15,4,12,8,16.  There must be a
multiple of 16 samples in the file.  After processing the first {\tt
  .abf} file it looks for another file with the same name except with
an additional upper or lower case 'A' before the {\tt .abf} extension
(if there are both, it uses the one with the upper case).  This file
must also have a 2048 byte header, the same channel order, and a
multiple of 16 samples, and the samples from this file are appended to
those from the first file.  The program then increments the ASCII
value of the extra letter and looks for another file in the same way.
This continues untils it fails to find the next file.

If no channel numbers are specified, all 16 channels will be extracted
to {\tt .chan} files.  The file names will be the same as {\tt
  whatever.abf}, except with the {\tt .abf} extension replaced with a
two-digit channel number followed by a {\tt .chan} extension.  Any
existing files with the same names will be overwritten without
warning.  If channel numbers are specfied (in a space-separated list),
only those channels will be extracted

\section{\tt tkss.tcl}
Using {\tt tkss.tcl}, the spike sorter can be run without using the
{\tt spikesort\_control\_panel}, simply by typing
\begin{verbatim}
tkss.tcl
\end{verbatim}
at the command line.  This brings up what the Spike Sorter User Guide
call the ``Spike Sorter Window''.  It will not dispatch spike sorts to
other computers like the control panel does, but you can run multiple
spikesorts in parallel on the same computer using the SPIKESORT button
to start them one at a time.  They will continue to run after you
quit.  You can kill all the spike sorts you are running by typing
\begin{verbatim}
killall spikesort
\end{verbatim}
at the command line.

You can also split a DataMAX file or an {\tt .abf} file into {\tt
  chan} files using the SPLIT button.  {\tt tkss.tcl} will not show you
raw data, but you can use {\tt waveform.tcl} for that.  It will not
show you the cluster ``diagram'' either, but you can get that by
typing
\begin{verbatim}
dot -Tps FILENAME.dot > FILENAME.ps;  gv -scale=2 FILENAME.ps
\end{verbatim}
And it will not bring up the {\tt .notes} file, but you can do that by
typing
\begin{verbatim}
gedit FILENAME.notes
\end{verbatim}
It will not invoke {\tt integrate, digitize, rpls, cpls, trachpls,
  muph}, or {\tt chan2hdt} for you, but those can all be run by hand.
And it will not merge the {\tt .edt} files for multiple channels into
a single {\tt .edt} file.  You would have to do that by changing unit
numbers with {\tt tmover} or {\tt sed} (figuring out yourself what
they should be), and by merging the resulting {\tt .edt} files with
the {\tt merge} utility.

But it does have all the functionality described under
``\textit{RESULTS}'' in the Spike Sorter User Guide.

\clearpage

\end{document}
